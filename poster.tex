\documentclass[plainsections,36pt]{sciposter}

\usepackage[brazil]{babel}		% Idioma do documento
\usepackage{xcolor}			    % Controle das cores
\usepackage[T1]{fontenc}		% Selecao de codigos de fonte.
\usepackage{graphicx}			% Inclusão de gráficos
\usepackage[utf8]{inputenc}		% Codificacao do documento (conversão automática dos acentos)
\usepackage{wallpaper}
\usepackage{wrapfig}
\usepackage{amsfonts,amssymb,amsmath,amsthm}
\usepackage{multicol}
\usepackage{anyfontsize}

\providecommand{\U}[1]{\protect\rule{.1in}{.1in}}
%EndMSIPreambleData
\setcounter{tocdepth}{4}

%opening
\definecolor{amarelo}{HTML}{FFCC00}
\definecolor{verde}{HTML}{006600}
\renewcommand{\papertype}{custom}
\setlength{\paperwidth}{90cm}
\setlength{\paperheight}{100cm}
\renewcommand{\setpspagesize}{
\ifthenelse{\equal{\orientation}{portrait}}{
\special{papersize=90cm,100cm}
}{\special{papersize=90cm,100cm}}}
\setlength{\topmargin}{0in}
\setlength{\headheight}{0in}
\setlength{\headsep}{0in}
\setlength{\textheight}{84cm}
\setlength{\textwidth}{75.5cm}
\setlength{\oddsidemargin}{2.4cm}
\setlength{\evensidemargin}{0cm}
\setlength{\parindent}{0.25in}
\setlength{\parskip}{0.25in}
\setlength{\pdfpagewidth}{90cm}
\setlength{\pdfpageheight}{100cm}
\newcommand\BackgroundPic{
\put(-82,65){
\parbox[b][\paperheight]{\paperwidth}{
\vfill
\centering
\includegraphics[width=\paperwidth,height=\paperheight,
keepaspectratio]{poster-xi-sic.jpg}
\vfill
}}}
\setlength{\columnseprule}{0pt}

\renewcommand{\titlesize}{\fontsize{84}{45}\selectfont }
\newcommand{\largo}{\fontsize{36}{40}\selectfont }
\makeatletter



% Redefine a função section
\renewcommand\section{\@startsection {section}{1}{\z@}{-1ex \@plus -0.5ex \@minus -.1ex}{0.8ex \@plus.1ex}{\normalfont\largo\bfseries}}
                        
\makeatother
\vspace{-1cm}

\def\thesection{}
%%% Título %%%
\title{\vspace{-0.2cm} \hspace{-15cm}\textcolor{amarelo}{\parbox{0.9\textwidth}{DIGITE O TÍTULO AQUI }}}
%%%
\usepackage{parskip}
\begin{document}

\AddToShipoutPicture{\BackgroundPic}
\makeatletter
\AddToShipoutPicture{%
            \setlength{\@tempdimb}{.5\paperwidth}%
            \setlength{\@tempdimc}{.5\paperheight}%
            \setlength{\unitlength}{1pt}%
            \put(\strip@pt\@tempdimb,\strip@pt\@tempdimc){%
                    }%
}
\makeatother
\maketitle
\mbox{}\vspace{6cm}

\begin{center}
\textbf{\textit{\large Autor 1, Autor 2}}\\
Centro de Engenharia, Modelagem e Ciências Sociais Aplicadas da UFABC\\
Av. dos Estados, 5001, Santo André, SP\\
\textit{\{email1, email2\}@ufabc.edu.br}
\end{center}

\vspace{1.2cm}
\begin{multicols}{2}
\paragraph{Resumo:} Este é um modelo de pôster que descreve o estilo a ser usado na confecção dos pôsteres para exibição no VIII Encontro de Iniciação Científica.

\textbf{Palavras-chave:} Modelo de pôster, VIII Encontro de Iniciação Científica.

\section*{Introdução}

Integer sapien sapien, sodales non sollicitudin sed, vestibulum volutpat odio. In sit amet interdum lectus.
Nulla venenatis, enim eu viverra posuere, orci justo auctor nulla, sed pellentesque nisl est sed mi.
Vivamus neque nisi, dignissim quis tincidunt et, tincidunt id est. Integer magna tortor, consequat et viverra eget,
condimentum et arcu. Fusce id odio non ipsum tempus blandit. Proin rhoncus dui ut turpis volutpat tempor tempus sapien varius. 
Proin rhoncus, massa a interdum vulputate, augue quam vehicula ante, ut luctus turpis lorem a nunc.
Mauris ac lorem enim, a vehicula arcu. Maecenas eu nunc urna, sit amet pretium neque.
Cras molestie lorem sed risus lacinia eu tempor nisi ultricies.
Pellentesque felis lorem, tristique ut vehicula sit amet, ultricies a ipsum.

\section*{Figuras, Tabelas e Equações}

Exemplo de figura dado pela Figura \ref{fig:exemplo}, exemplo de tabela dado pela Tabela \ref{tab:exemplo} e exemplo de equação dado pela equação (\ref{eq:exemplo}).

\begin{figure}[htb]
\centering
\includegraphics[width=6cm]{poster-xi-sic.jpg}
\caption{Exemplo de Figura. Fonte: XI SIC.}
\label{fig:exemplo}
\end{figure}

\begin{table}[!htb]
	\caption{Exemplo de Tabela}
	\centering
	\begin{tabular}{ccc}
	\hline
	Exemplo 1 & Exemplo 2 \\
	\hline
	10 & 10 \\
	\hline
	20 & 20 \\
	\hline
	\end{tabular}
	\label{tab:exemplo}
\end{table}

\begin{equation}
\sum_{i=1}^{N} \frac{x_i}{N} = \frac{x_1+x_2+\ldots+x_N}{N}, 	
\label{eq:exemplo}
\end{equation}


Integer sapien sapien, sodales non sollicitudin sed, vestibulum volutpat odio. In sit amet interdum lectus.
Nulla venenatis, enim eu viverra posuere, orci justo auctor nulla, sed pellentesque nisl est sed mi.

\section*{Conclusão}
Suspendisse eleifend ullamcorper dignissim. Quisque ut tortor non massa fringilla vehicula. 
Aenean ac metus orci, ut tincidunt leo. In ullamcorper ultricies lorem, nec rhoncus lorem sodales quis.
In pulvinar ornare imperdiet. Maecenas a venenatis eros. Curabitur ac metus at dolor blandit tempor.
Vestibulum sit amet est sit amet risus blandit placerat. Suspendisse ullamcorper eros nec dui cursus vestibulum. 

\section*{Como citar} As citações e referências devem ser feitas respeitando-se as normas definidas pela ABNT. Somente deverão ser apresentadas as referências citadas no referido pôster. Nas referências no texto citar o nome do autor e o ano da publicação (SANTOS, 2003) .

\section*{Referências}

SANTOS, Bruno A. Aspectos conceituais e arquiteturais para a criação de linhagens de agentes de software cognitivos e situados. 2003. 130f. Dissertação (Mestrado em Tecnologia – Manufatura Integrada por Computador) – Centro Federal de Educação Tecnológica de Minas Gerais, Belo Horizonte, 2003.

\section*{Agradecimentos}
Os alunos devem incluir em seus pôsteres uma seção de agradecimento com dizeres do tipo: Este trabalho foi financiado pelo Programa de Iniciação Científica da UFABC (PIC/UFABC) ou pelo CNPq (para bolsistas CNPq). 

\end{multicols}
\end{document}